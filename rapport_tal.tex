%%%%%%%%%%%%%%%%%%%%%%%%%%%%%%%%%%%%%%
 % Paramètres du document
%%%%%%%%%%%%%%%%%%%%%%%%%%%%%%%%%%%%%%

\documentclass[paper=a4]{article}
\usepackage[12pt]{extsizes} % taille de la police
\usepackage[margin=0.9in]{geometry} % taille de la marge

% choix de la langue et de l'option d'encodage
\usepackage[frenchb]{babel}
\usepackage[utf8]{inputenc}

% format de page avec le package "fancy header"
\usepackage{fancyhdr} % en-tête et pied personnalisés
\pagestyle{fancyplain} % toutes les pages sont conformes avec les en-têtes et pieds définis
\fancyhead{} % en-tête
\fancyfoot[L]{} % pied de gauche
\fancyfoot[C]{\thepage} % pied au centre
\fancyfoot[R]{}  % pied de droite
\renewcommand{\headrulewidth}{1pt}
\renewcommand{\footrulewidth}{1pt}
\setlength{\headheight}{13.6pt} % taille de l'en-tête

\usepackage{amsmath,amsfonts,amsthm} % package mathématique
\usepackage{graphicx} % package graphique
\usepackage{url} % hyperliens

\usepackage{multicol} % listes avec plusieurs colonnes

% page de garde
\usepackage{eso-pic}
\newcommand{\HRule}{\rule{\linewidth}{0.5mm}}
\newcommand{\blap}[1]{\vbox to 0pt{#1\vss}}
\newcommand\AtUpperLeftCorner[3]{%
  \put(\LenToUnit{#1},\LenToUnit{\dimexpr\paperheight-#2}){\blap{#3}}%
}
\newcommand\AtUpperRightCorner[3]{%
  \put(\LenToUnit{\dimexpr\paperwidth-#1},\LenToUnit{\dimexpr\paperheight-#2}){\blap{\llap{#3}}}%
}


%%%%%%%%%%%%%%%%%%%%%%%%%%%%%%%%%%%%%%
 % Page de garde
%%%%%%%%%%%%%%%%%%%%%%%%%%%%%%%%%%%%%%

\begin{document}

\title{Rapport TAL}
\author{Léo GALMANT\\Sacha LENARTOWICZ\\Maxime SAMSON}
\makeatletter

\begin{titlepage}

    \AddToShipoutPicture{
        \AtUpperLeftCorner{1.5cm}{1cm}{\includegraphics[width=4cm]{img/paris_saclay.jpg}}
        \AtUpperRightCorner{1.5cm}{1cm}{\includegraphics[width=6.5cm]{img/upsud.jpg}}
    }

    \begin{center}
        \vspace*{6.5cm}

        \Huge{\@title}
        \HRule
        \vspace*{0.5cm}

        \Large{\@author}

        \vspace*{7cm}
        \large{\today}
    \end{center}

\end{titlepage}
\ClearShipoutPicture


%%%%%%%%%%%%%%%%%%%%%%%%%%%%%%%%%%%%%%
 % Introduction
%%%%%%%%%%%%%%%%%%%%%%%%%%%%%%%%%%%%%%

\section*{Introduction}

Pour notre projet en Traitement Automatique des Langues (TAL), nous
avions le choix entre trois thèmes, et nous avons choisi de créer un
chatbot. Nous devions tout d'abord choisir quel type de chatbot
implémenter en choisissant un domaine de compétence. \\

Parmi les objectifs demandés, notre programme doit être capable de
converser au mieux avec un humain, de façon naturelle. De plus, les
réponses formulées doivent être grammaticalement correctes. Nous avions
également la tâche de permettre à notre chatbot de prendre en compte les
données entrées par l'utilisateur pour formuler ses réponses. \\

Nous avions le choix de la méthode sur laquelle construire notre
chatbot (heuristiques, motifs, mots-clefs, grammaires, réponses
génériques...). Nous devions également inclure des énoncés montrant que
le programme a compris et prend en compte ce que dit l'utilisateur.
\newpage % permet d'aller à la page suivante


%%%%%%%%%%%%%%%%%%%%%%%%%%%%%%%%%%%%%%
 % Partie 1 : nos choix
%%%%%%%%%%%%%%%%%%%%%%%%%%%%%%%%%%%%%%

\section{Nos choix}

Dans le contexte actuel des élections présidentielles, nous avons décidé
que le centre d'intérêt de notre programme serait la politique.
Nous vous laisserons le soin de découvrir si "Shabo" (nom que nous avons
donné au programme) est plutôt pro-capitaliste, ou au contraire s'il
possède un esprit plus communiste. \\

Nous avons également décidé de baser le fonctionnement sur une recherche
de mots-clefs. Nous avons trouvé ceci très adapté à notre objectif car le
vocabulaire que Shabo est capable de reconnaître est un vocabulaire
précis, et donc facilement détectable. \\

Enfin, nous avons choisi de travailler en anglais pour deux raisons. La
première, l'anglais est une langue plus facile à étudier que le
français, bien que nous maitrisons mieux le français, notre langue
natale. La deuxième raison est qu'il existe plus d'outils à notre
disposition pour l'anglais. \\


%%%%%%%%%%%%%%%%%%%%%%%%%%%%%%%%%%%%%%
 % Partie 2 : les outils utilisés
%%%%%%%%%%%%%%%%%%%%%%%%%%%%%%%%%%%%%%

\section{Les outils utilisés}

Dans l'objectif d'implémenter le plus efficacement notre programme, nous
devions utiliser le langage Python, pour lequel les cours de TAL que
nous avons reçus ont été d'une grande aide. En plus d'être un langage
particulièrement utile pour le traitement des chaînes de caractères, il
est possible, avec Python, d'inclure et d'utiliser des bibliothèques que
nous allons vous présenter.

 % 2.1 : PyEnchant
\subsection{PyEnchant}

PyEnchant est une librairie Python, permettant d'effectuer des
vérifications de langage, basée sur la librairie Enchant. Cet outil
nous a permis de vérifier une première chose (après la tokenisation de
la phrase, dont nous reparlerons dans la suite). On a pu exclure les
phrases composées de mots n'appartenant pas au dictionnaire anglais
britannique et anglais américain. Cette vérification est faite par la
fonction \textit{is\_english\_sentence()}. \\

Pour utiliser cette bibliothèque, vous devez l'installer à l'aide de la
commande suivante sous Linux : \textit{sudo apt-get install
python3-enchant}. Nous avons remarqué que cette vérification ne prenait
que peu de temps (quasi-immédiate), contrairement à nos craintes, et
nous avons donc choisi de la garder afin d'exclure les phrases
incorrectes pour faciliter l'exploitation des données entrées.

 % 2.2 : Wordnet
\subsection{Wordnet}

Wordnet peut être décrit comme étant une base de données lexicale portant
sur la langue anglaise. On retrouve cet outil dans la librairie Nltk.
Pour installer Wordnet, lancez un script Python contenant
\textit{nltk.download("wordnet", "chemin\_du\_programme")}. Ensuite,
lancez le script Python suivant :
\textit{nltk.data.path.append('chemin\_du\_programme')}. \\

Wordnet nous permet de trouver un ensemble de synonymes pour un mot
donné. Nous avons remarqué qu'une seule recherche prenait environ trois
secondes d'exécution. Nous avons donc tout d'abord prévu d'utiliser
ce module uniquement pour permettre d'enrichir nos propres
dictionnaires de mots-clefs. \\


%%%%%%%%%%%%%%%%%%%%%%%%%%%%%%%%%%%%%%
 % Partie 3 : Logique de fonctionnement
%%%%%%%%%%%%%%%%%%%%%%%%%%%%%%%%%%%%%%

\section{Logique de fonctionnement}

 % 3.1 : le pré-traitement
\subsection{Le pré-traitement}

Dans un premier temps, il a été nécessaire de pouvoir identifier les
mots entrés par l'utilisateur. Pour cela, nous avons repris la
"tokenisation" expérimentée lors des TP précédents, nous permettant
ainsi de repérer les éléments de ponctuation par la suite. Les
majuscules des mots sont également remplacées par des minuscules. \\

Comme expliqué dans la partie 2, nous avons vérifié que chaque token
correspondait soit à un élément de ponctuation, soit à un mot
appartenant au dictionnaire anglais britannique ou au dictionnaire
américain. Pour les phrases incorrectes, Shabo indiquera à l'utilisateur
qu'il a bien commis une erreur (réponse différente du cas où Shabo ne
comprend pas la phrase).

 % 3.2 : les affirmations
\subsection{Les affirmations}

Nous traitons différemment les deux types de requêtes données par
l'utilisateur : les questions et les affirmations. \\

Dans le cas d'une affirmation, nous utilisons un premier dictionnaire de
mots-clefs associés à la gauche, et un second correspondant au
vocabulaire de droite. Nous avons aussi défini, à l'aide de Wordnet, un
dictionnaire de mots positifs associant à chaque mot un niveau indiquant
la "force" du mot. Par exemple, "good" est de niveau 1, "great" de
niveau 2 et "wonderful" de niveau 3.
Nous avons fait de même pour composer un dictionnaire de mots négatifs. \\

Nous avons aussi cherché à reconnaître les adverbes, pouvant influencer
sur la "force" de l'affirmation. Les adverbes de base sont stockés dans
la liste \textit{MULTIPLICATORS}, mais nous utilisons une liste agrémentée
de synonymes toujours grâce à Wordnet, appelée
\textit{MULTIPLICATORS\_FINAL}. \\

Nous nous servons de ces différents éléments pour définir si, dans sa
globalité, une phrase affirmative va contrarier ou au contraire flatter
Shabo. En fonction des différents niveaux enregistrés, Shabo va
définir son humeur (\textit{humor} dans le code) et va associer un degré
de réponses parmi sept niveaux (trois négatifs, un neutre et trois
positifs). L'ensemble du traitement s'effectue dans la fonction
\textit{politicalArguments()}. \\

Vous trouverez ici la liste des mots considérés de gauche par Shabo :
\begin{multicols}{4} % permet de faire une liste pucée multi-colonnes
    \begin{itemize}
        \item left
        \item communism
        \item leftist
        \item ussr
        \item gulag
        \item gulags
        \item healthcare
        \item cuba
        \item socialism
        \item communists
        \item comrade
        \item socialists
        \item equality
        \item anarchism
        \item anarchists
        \item collectivism
        \item collectivists
        \item marx
        \item staline
        \item lenine
        \item stalin
        \item lenin
        \item kropotkine
        \item proudhon
        \item mao
        \item castro
        \item che
    \end{itemize}
\end{multicols}

Liste des mots de droite :
\begin{multicols}{4} % liste pucée multi-colonnes
    \begin{itemize}
        \item capitalism
        \item bourgeois
        \item bourgeoisie
        \item shareholder
        \item shareholders
        \item property
        \item boss
        \item CEO
        \item company
        \item companies
        \item usa
        \item america
        \item united states
    \end{itemize}
\end{multicols}

Liste des mots positifs :
\begin{multicols}{4} % liste pucée multi-colonnes
    \begin{itemize}
        \item good
        \item like
        \item acceptable
        \item valuable
        \item positive
        \item satisfactory
        \item fine
        \item great
        \item marvelous
        \item admirable
        \item amazing
        \item excellent
        \item exceptional
        \item wonderful
    \end{itemize}
\end{multicols}

Liste des mots négatifs :
\begin{multicols}{4} % liste pucée multi-colonnes
    \begin{itemize}
        \item bad
        \item wrong
        \item indecent
        \item odious
        \item despicable
        \item immoral
        \item hateful
        \item repugnant
        \item evil
    \end{itemize}
\end{multicols}


Parmi les affirmations, Shabo a également un comportement par défaut
aux phrases de salutation. De même, pour quitter le programme, vous
pouvez, en plus du Ctrl+C sous Linux, dire "bye" à Shabo, et le
programme se fermera.

 % 3.3 : les questions
\subsection{Les questions}

Pour la gestion des différentes questions, nous réutilisons les
dictionnaires liés aux vocabulaires de droite et de gauche. \\

Shabo est capable de reconnaître si la question posée est une question
personnelle. Le programme va alors répondre, suivant les réponses
prédéfinies associées au mot-clef reconnu. Vous pouvez demander à Shabo
son nom, son âge, s'il est humain et s'il va bien.\\

Pour chacun des mots de la question, Shabo vérifie s'il est dans le
dictionnaire de gauche ou de droite. Dans le cas où le mot est dans le
dictionnaire de réponses, il affiche la
réponse associée, contenant une définition du mot et l'avis que porte
le chatbot sur celui-ci. Sinon, si le mot appartient au vocabulaire de gauche
ou de droite, on incrémente un score, indiquant si la phrase est plus
composée de mots de gauche ou de droite. Si aucun mot n'appartient au
dictionnaire de réponse, alors une réponse par défaut, basée sur le
score de la question, est envoyée à l'utilisateur. Cette partie est
traitée dans la fonction \textit{politicalQuestion()}. \\

Voici une liste de mots pour lesquels Shabo est capable de répondre
convenablement :
\begin{multicols}{4} % liste pucée multi-colonnes
    \begin{itemize}
        \item left
        \item communism
        \item leftist
        \item ussr
        \item gulag
        \item gulags
        \item healthcare
        \item cuba
        \item socialism
        \item communists
        \item socialists
        \item equality
        \item anarchism
        \item anarchists
        \item collectivism
        \item collectivists
        \item marx
        \item lenin
        \item kropotkine
        \item che
        \item capitalism
        \item bourgeois
        \item bourgeoisie
        \item shareholder
        \item shareholders
        \item property
        \item CEO
        \item company
        \item companies
        \item usa
    \end{itemize}
\end{multicols}


%%%%%%%%%%%%%%%%%%%%%%%%%%%%%%%%%%%%%%
 % Partie 4 : Améliorations envisagées
%%%%%%%%%%%%%%%%%%%%%%%%%%%%%%%%%%%%%%

\section{Améliorations envisagées}

Comme dans de nombreux domaines, le projet consistant en la création
d'un chatbot n'est jamais terminé, et ceci est vrai même lorsque l'on
parle d'équipes importantes. C'est donc particulièrement le cas pour
notre groupe composé de seulement trois étudiants.

 % 4.1 : améliorations sur les affirmations
\subsection{Les améliorations sur les affirmations}

Concernant cette première partie, nous envisageons dans un premier temps
de modifier notre façon de répondre. Pour le moment, Shabo va considérer
une sorte de moyenne sur la totalité de la phrase, et répondre en
conséquence. Nous prévoyons de pouvoir diviser la phrase en plusieurs
parties, et ainsi de pouvoir répondre pour chacun des points mentionnés
par l'utilisateur. De plus, nous espérons pouvoir permettre à notre
chatbot de réagir sur les mots-clefs eux-mêmes, et pas seulement sur
leur orientation droite/gauche. \\

Nous avons commencé à définir un vocabulaire suffisamment large pour
pouvoir tester Shabo. Cependant, nous avons conscience que ce
vocabulaire ne représente qu'une partie des affirmations politiques dont
nous avons potentiellement envie de parler avec Shabo. Un enrichissement
du vocabulaire s'avère donc nécessaire. \\

 % 4.2 : améliorations sur les questions
\subsection{Les améliorations sur les questions}

Afin de ne pas seulement donner une définition des différents mots-clefs
reconnaissables, nous imaginons pouvoir différencier les différents
types de questions, entre les "yes or no questions" et les questions
ouvertes. On pourrait également adapter, pour les questions ouvertes,
le type de réponse en fonction du mot interrogatif en début de phrase. \\

Comme pour les affirmations, nous pouvons améliorer Shabo en continuant
notre travail sur l'enrichissement du vocabulaire, tout en restant dans
le contexte politique. Il en va de même pour les questions personnelles.

 % 4.3 : autres améliorations
\subsection{Autres améliorations}

Actuellement, les noms propres ne sont pas reconnus comme tels. Or, cela peut être
utile, car l'utilisateur peut avoir envie de demander ce que Shabo pense
de tel ou tel acteur politique. Il faudra pour cela ne pas supprimer la
majuscule lors de la tokenisation, ou alors ajouter un marqueur
indiquant que le mot est un nom propre. \\

Pour rendre le chatbot plus humain, nous aurions également pu
déterminer, pour la même phrase entrée, plusieurs réponses équivalentes.
Il aurait ensuite été aisé de sélectionner une réponse, de façon
aléatoire et en éliminant les réponses déjà renvoyées. \\

Nous imaginons également ne plus renvoyer des réponses toutes faites,
mais de permettre à Shabo, dans certains cas, de pouvoir créer ses
réponses respectant une structure donnée. \\

Enfin, nous n'enregistrons pas d'information portant sur les données
entrées par l'utilisateur. On envisage de retenir des informations
personnelles, comme le nom, le sexe de l'utilisateur, mais aussi son
avis sur certaines questions, pouvant modifier le ton utilisé par Shabo
pour répondre (plus ou moins agressif par exemple). \\


%%%%%%%%%%%%%%%%%%%%%%%%%%%%%%%%%%%%%%
 % Partie 5 : Répartition des tâches
%%%%%%%%%%%%%%%%%%%%%%%%%%%%%%%%%%%%%%

\section{Répartition des tâches}

Pour ce projet, nous avons commencé par travailler tous les trois
ensemble, afin de choisir les lignes directrices sur lesquelles nous
nous sommes orienté par la suite. Ainsi, nous avons pris, par exemple,
la décision de vérifier si une phrase est correcte, de commencer à
vérifier si la phrase de l'utilisateur est une affirmation ou une
question, et nous avons pris le choix de nous orienter sur une recherche
des mots-clefs. \\

Pour la deuxième moitié du projet, nous avons réparti les tâches comme
suivant : Léo GALMANT s'est occupé de la gestion des affirmations,
Maxime SAMSON des différents types de questions, et Sacha LENARTOWICZ de
la rédaction de ce rapport, tout en continuant de suggérer certains
choix pour améliorer Shabo. La rédaction du rapport, imposée en LaTeX,
fut l'occasion de découvrir ce langage que nous ne connaissions que de
nom. \\


%%%%%%%%%%%%%%%%%%%%%%%%%%%%%%%%%%%%%%
 % Conclusion
%%%%%%%%%%%%%%%%%%%%%%%%%%%%%%%%%%%%%%

\section*{Conclusion}

Le projet que nous avons choisi nous a permis de mettre en application
les différents TP réalisés en cours de TAL. Cela nous a aussi permis de
travailler en Python, et de nous rendre compte des avantages que
fournissait ce langage pour travailler sur le langage. \\

Bien que notre chatbot soit, de loin, moins perfectionné que ceux qui
commencent à envahir notre quotidien, nous avons constaté les
difficultés que consistent sa mise en place. Nous avons aussi pu étudier
les différentes techniques utilisables pour l'implémentation d'un tel
outil.

\end{document}
